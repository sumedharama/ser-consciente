\part{Reflexão}

\chapter{A Necessidade de Sabedoria no Mundo}

Estamos aqui todos com um interesse comum. Em vez de esta ser uma sala
cheia de indivíduos em que cada um segue os seus próprios pontos de
vista e opiniões, esta noite estamos todos aqui devido ao interesse
comum na prática do \emph{Dhamma}. Quando todas estas pessoas se
juntam a um Domingo à noite, começamos a ver o potencial da existência
humana, uma sociedade baseada no interesse comum da verdade. Mergulhamos
no \emph{Dhamma}. Aquilo que surge, cessa, e na sua cessação está a paz.
Então quando começamos a abrir mão dos nossos hábitos e apegos ao mundo
fenoménico, condicionado, começamos a realizar a totalidade e a unidade
da mente.

Isto é uma reflexão muito importante para estes tempos, nos quais
acontecem tantas disputas e guerras devido ao facto de as pessoas não
conseguirem concordar em nada. Os chineses contra os russos, os
americanos contra os soviéticos, e por aí fora. Acerca do quê? Estão a
lutar pelo quê? Pelas percepções que cada um deles tem do mundo. `Esta é
a \emph{minha} terra e eu quero \emph{isto} desta forma. Eu quero
\emph{este} tipo de governo e \emph{este} tipo de sistema político e
económico' e por aí adiante, até ao ponto em que esquartejamos e
torturamos até termos destruído a terra que estávamos a tentar libertar,
escravizando ou confundindo todas as pessoas que tentávamos libertar.
Porquê? Porque não entendermos as coisas como elas realmente são.

O caminho do \emph{Dhamma} é o de observar a natureza e de harmonizarmos
as nossas vidas com as forças naturais. Na civilização europeia nunca
olhámos realmente para o mundo dessa maneira. Idealizámo"-lo. Se tudo
fosse um ideal então deveria ser de determinada maneira, e quando pura e
simplesmente nos apegamos a ideais acabamos por fazer o que temos
estado a fazer ao nosso planeta até agora, poluindo"-o, chegando à sua
quase total destruição por não percebemos as limitações que nos são
impostas através das condições da Terra. Em todas as coisas desta
natureza, por vezes, temos de aprender pela via árdua, fazendo tudo
errado até ao ponto de termos um grande caos. Confiemos que esta
situação ainda tenha solução.

Neste momento, no mosteiro, os monges e as monjas estão a praticar o
\emph{Dhamma} diligentemente. Durante todo o mês de Janeiro não estamos
sequer a falar, mas a dedicar as nossas vidas e a oferecer as bênçãos da
nossa prática para o bem"-estar de todos os seres viventes. Todo este mês
é uma contínua prece e oferta desta comunidade para o bem"-estar de todos
os seres viventes. É um tempo dedicado exclusivamente à realização da
verdade, observando e escutando as coisas como elas são, um tempo para
nos abstermos dos actos ou estados de ânimo egoístas, desistindo de tudo
isso pelo bem"-estar de todos os seres vivos. Este tipo de dedicação e
`sacrifício', em direcção à verdade, é algo para todos reflectirem. É um
indicador que aponta em direcção à realização da verdade na nossa
própria vida, em vez de apenas vivermos de um modo automático e habitual
seguindo a conveniência das condições do momento. Isto também é uma
reflexão para os outros: desistir de actividades imorais, egoístas ou
cruéis para se transformar em alguém que visa a impecabilidade,
generosidade, moralidade e acção compassiva no mundo. Se não fizermos
isto ficamos numa situação completamente desastrosa. Se ninguém estiver
disposto a usar a sua vida para nada mais do que concessões egoístas,
pode"-se mesmo vir a acabar com tudo, e nada valerá a pena.

Este país (Inglaterra) é um país generoso e benevolente, mas nós
simplesmente tomamo"-lo como garantido e exploramo"-lo para simplesmente 
obter aquilo que queremos. Não pensamos muito em lhe dar algo. Exigimos 
muito, queremos que o governo faça tudo de bom para nós e depois 
criticamo"-lo quando não o pode fazer. Nos dias de hoje encontram"-se 
indivíduos egoístas a viver a vida à sua maneira, sem reflectir sabiamente, 
vivendo de uma forma que faz com que não sejam uma bênção para a sociedade. 
Como seres humanos podemos transformar as nossas vidas em grandes bênçãos; 
ou podemos tornar"-nos uma praga para o planeta, utilizando os recursos 
da Terra para lucro pessoal e obtendo tanto quanto pudermos para nós 
próprios, para `mim' e para o `meu'.

Na prática do \emph{Dhamma} a noção de `mim' e `meu' começa a
desvanecer"-se -- essa noção de `mim' e `meu' como esta pequena criatura
aqui sentada que tem uma boca e que tem de comer. Se somente seguir os
desejos do meu corpo e as minhas emoções, torno"-me numa pequena criatura
gananciosa e egoísta. Mas quando reflicto na natureza da minha condição
física e quão sabiamente pode ser utilizada nesta vida para o bem"-estar
de todos os seres vivos, vejo que pode tornar"-se numa bênção. (Não que
cada um se veja como uma bênção, pois o `Eu sou uma bênção' é apenas um
outro tipo de conceito, caso nos comecemos a apegar a essa ideia!) Assim
vivemos efectivamente cada dia de maneira a que a nossa vida traga
alegria, compaixão, gentileza ou que pelo menos não cause confusão
ou miséria desnecessárias. O mínimo que podemos fazer é manter os Cinco
Preceitos\footnote{%
  Os Cinco Preceitos são os preceitos éticos básicos a serem observados
  por cada praticante budista.}
e dessa forma os nossos corpos e as nossas palavras não estão
a ser usados para a corrupção, crueldade e exploração do nosso planeta.
Isso é pedir muito de cada um de vós? É assim tão extraordinário deixar
de fazer só o que nos apetece em cada momento, de maneira a sermos pelo
menos um pouco mais cuidadosos e responsáveis por aquilo que fazemos e
dizemos? Todos nós podemos tentar ajudar, ser generosos e amáveis e ter
consideração para com os outros seres com quem temos de partilhar o
planeta. Todos nós podemos sabiamente investigar e compreender as
limitações sob as quais nos encontramos, de forma a já não nos iludirmos
com o mundo das sensações. É por isto que meditamos. Para um monge ou
uma monja, isto é uma forma de vida, um sacrifício dos nossos desejos
particulares e caprichos, para o bem"-estar da comunidade, do Saṅgha.

Se começar a pensar sobre mim próprio e sobre o que eu quero, esqueço"-me
do resto de vós, pois o que particularmente quero neste momento poderá
não ser bom para vós. Mas quando uso o refúgio no Saṅgha como meu guia,
então o bem"-estar do \emph{Saṅgha} é a minha alegria e abro mão dos meus
caprichos
pessoais para o bem"-estar do \emph{Saṅgha}. É por isso que os monges e
as monjas têm de rapar as cabeças e viver sob uma disciplina
estabelecida pelo \emph{Buddha}. Isto é uma maneira de nos treinarmos a
ter como atitude de vida o `abrir mão do `eu'', uma maneira de estar que
não nos traz vergonha, culpa ou medo. A noção de uma individualidade
disruptiva é perdida, por já não estarmos determinados a ser
independentes do resto, ou a dominar, mas sim a harmonizar e a vivermos
para o bem"-estar de todos os seres e não para o nosso bem"-estar pessoal.

A comunidade leiga tem a oportunidade de participar nisto. Os monges e
as monjas dependem da comunidade leiga para a sua sobrevivência básica,
logo é importante para a comunidade leiga tomar essa responsabilidade.
Isto tira as pessoas leigas dos seus problemas e obsessões particulares,
pois quando tiram o seu tempo para virem aqui para dar, ajudar, praticar
meditação e ouvir o \emph{Dhamma}, encontram"-se a eles próprios imersos
na unidade da verdade. Podemos estar aqui todos juntos sem inveja,
ciúmes, medo, dúvida, ganância ou luxúria devido a estarmos unidos no
propósito da realização da verdade. Façam essa a intenção da vossa vida,
não a desperdicem com buscas tolas!

Esta verdade, pode ser chamada de muitas maneiras. As religiões tentam
de alguma forma passar essa verdade -- através de conceitos e doutrinas
-- mas nós esquecemo"-nos para o que serve a religião. Nos últimos cem
anos a sociedade tem vindo a seguir a ciência materialista, o pensamento
racional e o idealismo baseado na nossa capacidade de conceber sistemas
políticos e económicos e, ainda assim, não conseguimos que (eles)
funcionem, ou conseguimos? Na realidade, não podemos criar uma democracia
ou um verdadeiro comunismo ou um verdadeiro socialismo -- não
conseguimos criá"-los por ainda estarmos iludidos com a noção do `eu'.
Por isso tudo acaba em tirania, egoísmo, medo e suspeita.

Desta forma, a presente situação é o resultado de não se compreender a
realidade tal qual ela é, e esta é uma época em que cada um de nós deve,
se estivermos realmente interessados no que podemos fazer, transformar a
própria vida em algo que valha a pena. E como fazê"-lo?

Primeiro precisamos de admitir as motivações e concessões egoístas
(resultantes da imaturidade emocional) que temos, para as conhecermos e
nos desapegarmos delas; abrir a mente para as verdade, estar alerta. A
nossa prática de \emph{ānāpānasati} já é um começo, não é? Não é apenas
mais um hábito ou passa"-tempo que desenvolvemos para nos mantermos
ocupados. É um meio de nos esforçarmos para observar, concentrar e ser
como a respiração é. Pode"-se em vez disso passar bastante tempo a ver
televisão, a ir aos bares e a fazer todo o tipo de coisas que não são 
muito úteis -- de alguma forma isso parece mais importante do que
passar o tempo a observar a nossa própria respiração, não é? Vemos as
notícias da televisão onde pessoas são massacradas no Líbano -- de certa
forma isso parece mais importante do que simplesmente sentarmo"-nos a
observar a nossa inspiração e expiração. Mas isto resulta do facto da
mente não compreender a forma como as coisas são; por isso estamos
dispostos a observar as sombras no ecrã da televisão e tudo o que este
transmite sobre a miséria que resulta da ganância, do ódio e da
estupidez, levada a cabo de forma desprezível. Não seria muito melhor
passar esse tempo com o corpo tal como ele está nesse momento? Seria
melhor respeitar este ser físico, aprendendo a não explorá"-lo e
abusá"-lo, arrependendo"-nos depois quando ele já não nos dá a felicidade
desejada.

Na vida monástica não temos televisão porque dedicamos as nossas vidas a
fazer coisas mais úteis, como observar a respiração e andar para cima e
para baixo nos caminhos de meditação na floresta. Os vizinhos pensam que
somos meio malucos. Todos os dias eles vêem pessoas a sair embrulhadas
em cobertores, a andar para cima e para baixo. `O que é que eles estão a
fazer? Devem ser malucos!'

Há algumas semanas atrás houve caça à raposa. As matilhas estavam a
perseguir raposas nos nossos bosques (estavam a fazer algo bastante útil
e benéfico para todos os seres vivos!). Sessenta cães e toda aquela
gente crescida a correr atrás de uma pequena pobre raposa. Seria melhor
passarem esse tempo a andar para cima e para baixo no caminho de
meditação, não era? Melhor para a raposa, para os cães, para Hammer Wood
e para os caçadores de raposas. Mas as pessoas em West Sussex pensam que
eles são normais. Eles são os normais e nós somos os maluquinhos. Quando
observamos a nossa respiração e andamos para cima e para baixo no
caminho de meditação na floresta, pelo menos não
estamos a aterrorizar raposas! Como é que nos sentiríamos se
sessenta cães estivessem a perseguir"-nos. Imaginem o que seria do nosso
coração se tivéssemos uma matilha de sessenta cães a correr atrás de nós
e pessoas a cavalo a dizerem"-lhes para nos apanharem. Quando realmente
reflectimos nisto vêmos o quão desumano é. Ainda assim isso é
considerado normal ou até mesmo desejável nesta parte de
Inglaterra. Devido ao facto de o ser humano não ter tempo para
reflectir, pode ser vítima de hábitos, sendo apanhado em hábitos e
desejos. Se realmente reflectíssemos sobre a caça à raposa, não o
faríamos. Se tivermos alguma inteligência e realmente pensarmos nisso,
não o desejaremos fazer. No entanto, com coisas simples como andar para
cima e para baixo num caminho de meditação numa floresta, e observar a
nossa respiração, tornamo"-nos conscientes e muito mais sensiveis. A
verdade começa a revelar"-se através das simples e aparentemente
insignificantes práticas que fazemos, tal como mantermos os Cinco
Preceitos: viver no Dhamma torna"-se um campo fértil de bênçãos para o mundo.

Quando começamos a reflectir na forma como as coisas são e nos lembramos
dos momentos em que a nossa vida esteve realmente em perigo, sabemos
efectivamente o quão horrível esse sentimento é. É uma experiência absolutamente
aterradora. Se reflectissemos nisso, nunca iríamos querer que outra
criatura passasse por essa experiência. De maneira alguma iríamos
intencionalmente sujeitar outra criatura a esse terror. Se não
reflectirmos, pensamos que as raposas e os peixes não importam. Eles
existem para o nosso prazer -- é algo para fazer ao domingo à tarde.
Lembro"-me de uma mulher que veio ver"-me e que estava bastante aborrecida
por termos comprado a Hammer Pond\footnote{%
  Ao fazer parte de um mosteiro Budista, Hammer Wood e Pond tornaram"-se,
  é claro, santuários da vida selvagem.}.
Ela disse `Sabe, tenho tanta paz; eu
não venho aqui para pescar, eu venho aqui pela paz de estar aqui.' Ela
passava todos os domingos a apanhar peixes, simplesmente para estar em
paz. Ela parecia ser bastante saudável, era um pouco rechonchuda, não
estava a morrer à fome; não precisava realmente de pescar para
sobreviver. Eu disse: `Bom, se não precisa de pescar para sobreviver --
espero que tenha dinheiro suficiente para comprar peixe -- pode vir cá,
depois de termos comprado este lago, e simplesmente meditar. Não tem de
pescar.'

Ela não queria meditar! Continuou a falar, acerca dos coelhos que lhe
comiam as couves e que teve de pôr todo o tipo de coisas para matar os
coelhos, para que estes não lhe comessem as couves. Esta mulher nunca
reflectia em nada. Maltratava os coelhos por comerem as couves dela, mas
ela podia muito bem ir comprar couves a algum lugar. Mas os coelhos não.
Eles têm de fazer o melhor que podem, comendo as couves de alguém. Mas
ela nunca abriu realmente a mente para as coisas como elas são, para o
que é verdadeiramente atencioso e benéfico. Eu não diria que ela é uma
pessoa cruel ou sem coração, mas sim simplesmente uma mulher de classe média 
que nunca reflectiu sobre a natureza ou que nunca percebeu
o que é o \emph{Dhamma}. Por isso ela pensa que as couves estão lá para
ela e não para os coelhos, e que os peixes estão lá para que ela possa
passar uma pacífica tarde de domingo a torturá"-los.

É para esta capacidade de reflectir e observar que o \emph{Buddha}
aponta nos seus ensinamentos, como a forma de nos libertar- mos da
sujeição cega aos hábitos e convenções. É o caminho para nos libertarmos
da ilusão das condições sensoriais, através da sábia reflexão sobre a forma
como as coisas são. Começamos por nos observar, por observar o desejo ou
a aversão a algo, o entorpecimento ou a estupidez da mente. Não
estamos a procurar, escolher ou a tentar criar condições agradáveis para
o nosso prazer pessoal, mas estamos dispostos, de uma forma equânime,
a suportar condições desagradáveis ou até miseráveis, de modo a
compreendê"-las apenas como tal, e sermos capazes de as `deixar ir'.
Começamos a libertar"-nos do desejo de fugir das coisas que não gostamos.
Começamos também a ser muito mais cuidadosos acerca da maneira como
vivemos. Quando nos apercebemos desta realidade começamos a
verdadeiramente querer ser muito, muito cuidadosos com o que fazemos e
com o que dizemos. Já não temos qualquer intenção de viver às custas de
outrem. Já não sentimos que a nossa vida é muito mais importante que a
de qualquer outra pessoa. Começamos a sentir a liberdade e a leveza
dessa harmonia com a natureza em vez do peso de a explorar para proveito
pessoal. Quando se abre a mente para a verdade compreende"-se que não há
nada a temer. O que surge, irá passar, o que nasce, morre e é `não"-ser'
-- assim, a nossa noção de estarmos presos à identificação com este
corpo humano, desvanece"-se. Não nos vemos como uma entidade isolada,
alienada, perdida num misterioso e assustador universo. Não nos sentimos
amedrontados por tal, tentando encontrar uma pequena parte à qual nos
possamos agarrar, sentindo"-nos seguros pelo facto de estarmos em paz com
isso. Nesse momento fundimo"-nos com a verdade.

\chapter{Os Refúgios e os Preceitos}

`Tomar refúgio' e `praticar os Preceitos' é o que define uma
pessoa como praticante budista.

`Tomar refúgio' oferece uma perspectiva contínua na vida por estabelecer
uma conexão entre a nossa conduta e compreensão da vida, com as
qualidades de Buddha (sabedoria), Dhamma (verdade) e Sangha (virtude).
Os preceitos servem também como reflexão e definição de acções do ser
humano como um ente responsável.

Existe uma maneira formal de pedir os Refúgios e os Preceitos a um monge
ou a uma monja.

{\setlength{\parindent}{0pt}%
\fontsize{10}{16}\selectfont%

\begin{instruction}
  Após fazer três vénias, com as palmas\\
  das mão unidas em añjali, recita-se o pedido:
\end{instruction}

Mayaṃ bhante tisaraṇena saha pañca sīlāni yācāma\\
Dutiyampi mayaṃ bhante tisaraṇena saha pañca sīlāni yācāma\\
Tatiyampi mayaṃ bhante tisaraṇena saha pañca sīlāni yācāma

\begin{english}
  Pedimos, Venerável Mestre,\\
  \vin os Três Refúgios e os Cinco Preceitos.

  Pela segunda vez, pedimos, Venerável Mestre,\\
  \vin os Três Refúgios e os Cinco Preceitos.

  Pela terceira vez, pedimos, Venerável Mestre,\\
  \vin os Três Refúgios e os Cinco Preceitos.
\end{english}

\section{Os Três Refúgios}

\begin{instruction}
  Repetir, depois de o líder ter\\
  cantado as primeiras três linhas
\end{instruction}

Namo tassa bhagavato arahato sammāsambuddhassa\\
Namo tassa bhagavato arahato sammāsambuddhassa\\
Namo tassa bhagavato arahato sammāsambuddhassa

\begin{english}
  Homenagem ao Excelso, Nobre e Perfeitamente Iluminado.\\
  Homenagem ao Excelso, Nobre e Perfeitamente Iluminado.\\
  Homenagem ao Excelso, Nobre e Perfeitamente Iluminado.
\end{english}

Buddhaṃ saraṇaṃ gacchāmi\\
Dhammaṃ saraṇaṃ gacchāmi\\
Saṅghaṃ saraṇaṃ gacchāmi

\begin{english}
  Tenho o Buddha como refúgio.\\
  Tenho o Dhamma como refúgio.\\
  Tenho o Saṅgha como refúgio.
\end{english}

Dutiyampi buddhaṃ saraṇaṃ gacchāmi\\
Dutiyampi dhammaṃ saraṇaṃ gacchāmi\\
Dutiyampi saṅghaṃ saraṇaṃ gacchāmi

\begin{english}
  Pela segunda vez, tenho o Buddha como refúgio.\\
  Pela segunda vez, tenho o Dhamma como refúgio.\\
  Pela segunda vez, tenho o Saṅgha como refúgio.
\end{english}

Tatiyampi buddhaṃ saraṇaṃ gacchāmi\\
Tatiyampi dhammaṃ saraṇaṃ gacchāmi\\
Tatiyampi saṅghaṃ saraṇaṃ gacchāmi

\begin{english}
  Pela terceira vez, tenho o Buddha como refúgio.\\
  Pela terceira vez, tenho o Dhamma como refúgio.\\
  Pela terceira vez, tenho o Saṅgha como refúgio.
\end{english}

\begin{instruction}
  Líder:
\end{instruction}

[Tisaraṇa-gamanaṃ niṭṭhitaṃ]

\begin{english}
  Fica assim completo o Triplo Refúgio.
\end{english}

\begin{instruction}
  Resposta:
\end{instruction}

Āma bhante

\begin{english}
  Sim, Venerável Mestre.
\end{english}

\section{Os Cinco Preceitos}

\begin{instruction}
  Repetir cada preceito depois do líder
\end{instruction}

\begin{precept}
  \setcounter{enumi}{0}
  \item Pāṇātipātā veramaṇī sikkhāpadaṃ samādiyāmi
\end{precept}

\begin{english}
  Observo o preceito de me abster de matar qualquer criatura viva.
\end{english}

\begin{precept}
  \setcounter{enumi}{1}
  \item Adinnādānā veramaṇī sikkhāpadaṃ samādiyāmi
\end{precept}

\begin{english}
  Observo o preceito de me abster de tirar aquilo que não me for oferecido.
\end{english}

\begin{precept}
  \setcounter{enumi}{2}
  \item Kāmesu micchācārā veramaṇī sikkhāpadaṃ samādiyāmi
\end{precept}

\begin{english}
  Observo o preceito de me abster de ter uma conduta sexual imprórpia.
\end{english}

\begin{precept}
  \setcounter{enumi}{3}
  \item Musāvādā veramaṇī sikkhāpadaṃ samādiyāmi
\end{precept}

\begin{english}
  Observo o preceito de me abster de mentir.
\end{english}

\begin{precept}
  \setcounter{enumi}{4}
  \item Surāmeraya-majja-pamādaṭṭhānā veramaṇī sikkhāpadaṃ samādiyāmi
\end{precept}

\begin{english}
  Observo o preceito de me abster de consumir bebidas\\
  e drogas intoxicantes que deturpem a mente.
\end{english}

\begin{instruction}
  Líder:
\end{instruction}

[Imāni pañca sikkhāpadāni\\
Sīlena sugatiṃ yanti\\
Sīlena bhogasampadā\\
Sīlena nibbutiṃ yanti\\
Tasmā sīlaṃ visodhaye]

\begin{english}
  Estes são os Cinco Preceitos;\\
  A virtude é fonte de felicidade,\\
  A virtude é fonte de verdadeira riqueza,\\
  A virtude é fonte de paz --\\
  Que a virtude seja assim purificada.
\end{english}

\begin{instruction}
  Resposta:
\end{instruction}

Sādhu, sādhu, sādhu

\begin{instruction}
  Fazer três vénias.
\end{instruction}

}
