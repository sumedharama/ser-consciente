\chapter{Antes de Começar}

A maior parte destas instruções podem ser levadas a cabo quer estejamos
sentados, em pé ou a andar. Contudo, estar consciente da respiração
(\emph{ānāpānasati}) tal como apresentado nos primeiros capítulos é algo
geralmente realizado na postura de sentado, uma vez que esse trabalho é
potenciado quando realizado num estado físico de quietude e
estabilidade. Para este estado a ênfase é sentarmo"-nos de forma a que a
coluna esteja erguida mas não em esforço, com o pescoço alinhado com a
coluna e a cabeça em equilíbrio de forma a não pender para a frente. É
de opinião geral que a postura de lótus de pernas cruzadas (sentados
numa almofada ou num tapete, com um ou ambos os pés colocados na coxa
oposta, com as plantas dos pés para cima) confere um equilíbrio ideal
entre esforço e estabilidade -- após alguns meses de prática. É bom
treinarmo"-nos gentilmente nesta direcção, um pouco de cada vez. Se esta
postura for muito difícil pode ser usada uma cadeira de costas direitas.

Após ter"-se alcançado estabilidade e um certo equilíbrio físico, a cara
e os braços devem ficar descontraidos, com as mãos a descansar no colo,
uma na palma da outra. Deixem as pálpebras fecharem"-se, relaxem a
mente\ldots{} escolham o objecto de meditação.

\emph{Jongrom}, uma palavra Tailandesa derivada do Pāli (linguagem das
escrituras) `\emph{caṅkama}' significa passear para a frente e para trás
num caminho a direito. O caminho deve ser medido, sendo o ideal vinte a
trinta passos entre dois objectos claramente
identificáveis de modo a não ter que contar os passos enquanto
pratica o \emph{jongrom}. As mãos deverão estar unidas de uma forma
suave, à frente ou atrás do corpo, com os braços relaxados. O olhar
deverá estar direccionado no caminho sem se focar nele, a uma distância
de aproximadamente dez passos à frente, não para observar nada em
particular mas para manter o ângulo mais confortável para o pescoço. O
caminhar começa então com uma postura correcta e quando se chega ao fim
do percurso devemos permanecer imóveis por um período que pode durar uma
ou duas respirações. Então, conscientemente, viramo"-nos e recomeçamos a
andar no sentido contrário.

