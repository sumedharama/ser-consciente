\chapter{Sobre o Autor}

Luang Pô Sumedho nasceu em Seattle, Washington, em 1934. Estudou Chinês e
História na Universidade de Washington. Depois de servir quatro anos como
médico-assistente na Marinha dos Estados Unidos, regressa à Universidade e
completa um bacharelato em Estudos do Extremo-Oriente.

Os estudos introduzem-no ao Budismo, enquanto o período de serviço na Marinha
leva-o a entrar em contacto com a `Sociedade Budista do Japão'. Em 1961,
ingressa novamente aos estudos, para um mestrado em Estudos do Sul-Ásiático, na
Universidade da Califórnia, Bekerley, onde se graduou em 1963.

Em 1966, decide viajar até à Tailândia para praticar meditação em Wat Mahathat,
Bangkok. Não muito depois, toma ordenação como monge em Nong Khai, até receber 
ordenação completa em 1967.

Segue-se ano de prática solitária, o qual, apesar de frutuoso, mostrou-lhe a
necessidade de um professor que pudesse guiá-lo mais activamente. Um encontro
furtuito com um monge em visita, levou-o ao encontro de Luang Pô Chah,
tornando-se seu discípulo e ficando sob sua orientação íntima durante dez anos.

Em 1975, Luang Pô Chah encarrega-o de liderar uma pequena comunidade de monges,
fundando assim um `\mbox{Mosteiro}~da Tradição da Floresta' para monges do Ocidente,
Wat Pah Nanachat, onde os ocidentais podem treinar em inglês. No ano de 1976,
Luang Pô Sumedho realiza uma viagem aos Estados Unidos da América para visita os
pais, fazendo escala em Inglaterra, onde foi convidado a ficar num pequeno
Centro Budista em Londres. No ano seguinte ficou aí a residir, na companhia de
mais três monges.

Desde então fundou vários mosteiros no Reino Unido e restante Europa, tendo sido
durante muitos anos o Abade do Mosteiro Budista Amarāvatī, em Inglaterra.

Luang Pô Sumedho inspirou mais de uma centena de aspirantes, de várias
nacionalidades, a seguirem a vida monástica. Estabeleceu quatro mosteiros em
Inglaterra, bem como vários outros em várias partes da Europa e restante mundo
Ocidental, incluindo Sumedhārāma, em Portugal.

No final de 2010, reformou-se como abade do Mosteiro Budista Amarāvatī, e passou
a residir na Tailândia. Contudo, regressou a Amarāvatī em Janeiro de 2021, onde
tem vindo a oferecer regularmente ensinamentos de Dhamma.
